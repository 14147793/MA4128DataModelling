%========================================================== %
Amalgamation or Linkage Rules
At the first step, when each object represents its own cluster, the distances between those objects are defined by the chosen distance measure. However, once several objects have been linked together, how do we determine the distances between those new clusters? In other words, we need a linkage or amalgamation rule to determine when two clusters are sufficiently similar to be linked together. There are various possibilities: for example, we could link two clusters together when any two objects in the two clusters are closer together than the respective linkage distance. Put another way, we use the "nearest neighbors" across clusters to determine the distances between clusters; this method is called single linkage. This rule produces "stringy" types of clusters, that is, clusters "chained together" by only single objects that happen to be close together. Alternatively, we may use the neighbors across clusters that are furthest away from each other; this method is called complete linkage. There are numerous other linkage rules such as these that have been proposed.
%=============================================================== %
Single linkage (nearest neighbor). As described above, in this method the distance between two clusters is determined by the distance of the two closest objects (nearest neighbors) in the different clusters. This rule will, in a sense, string objects together to form clusters, and the resulting clusters tend to represent long "chains."
Complete linkage (furthest neighbor). In this method, the distances between clusters are determined by the greatest distance between any two objects in the different clusters (i.e., by the "furthest neighbors"). This method usually performs quite well in cases when the objects actually form naturally distinct "clumps." If the clusters tend to be somehow elongated or of a "chain" type nature, then this method is inappropriate.
%=============================================================== %
Unweighted pair-group average. In this method, the distance between two clusters is calculated as the average distance between all pairs of objects in the two different clusters. This method is also very efficient when the objects form natural distinct "clumps," however, it performs equally well with elongated, "chain" type clusters. Note that in their book, Sneath and Sokal (1973) introduced the abbreviation UPGMA to refer to this method as unweighted pair-group method using arithmetic averages.
%================================================================ %
Weighted pair-group average. This method is identical to the unweighted pair-group average method, except that in the computations, the size of the respective clusters (i.e., the number of objects contained in them) is used as a weight. Thus, this method (rather than the previous method) should be used when the cluster sizes are suspected to be greatly uneven. Note that in their book, Sneath and Sokal (1973) introduced the abbreviation WPGMA to refer to this method as weighted pair-group method using arithmetic averages.
Unweighted pair-group centroid. The centroid of a cluster is the average point in the multidimensional space defined by the dimensions. In a sense, it is the center of gravity for the respective cluster. In this method, the distance between two clusters is determined as the difference between centroids. Sneath and Sokal (1973) use the abbreviation UPGMC to refer to this method as unweighted pair-group method using the centroid average.
%=============================================================== %
Weighted pair-group centroid (median). This method is identical to the previous one, except that weighting is introduced into the computations to take into consideration differences in cluster sizes (i.e., the number of objects contained in them). Thus, when there are (or we suspect there to be) considerable differences in cluster sizes, this method is preferable to the previous one. Sneath and Sokal (1973) use the abbreviation WPGMC to refer to this method as weighted pair-group method using the centroid average.
%=============================================================== %
Ward's method. This method is distinct from all other methods because it uses an analysis of variance approach to evaluate the distances between clusters. In short, this method attempts to minimize the Sum of Squares (SS) of any two (hypothetical) clusters that can be formed at each step. Refer to Ward (1963) for details concerning this method. In general, this method is regarded as very efficient, however, it tends to create clusters of small size.
For an overview of the other two methods of clustering, see Two-way Joining and k-Means Clustering.
To index


