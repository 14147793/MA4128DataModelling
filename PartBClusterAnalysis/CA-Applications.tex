\documentclass[a4paper,12pt]{report}
%%%%%%%%%%%%%%%%%%%%%%%%%%%%%%%%%%%%%%%%%%%%%%%%%%%%%%%%%%%%%%%%%%%%%%%%%%%%%%%%%%%%%%%%%%%%%%%%%%%%%%%%%%%%%%%%%%%%%%%%%%%%%%%%%%%%%%%%%%%%%%%%%%%%%%%%%%%%%%%%%%%%%%%%%%%%%%%%%%%%%%%%%%%%%%%%%%%%%%%%%%%%%%%%%%%%%%%%%%%%%%%%%%%%%%%%%%%%%%%%%%%%%%%%%%%%
\usepackage{eurosym}
\usepackage{vmargin}
\usepackage{amsmath}
\usepackage{graphics}
\usepackage{epsfig}
\usepackage{subfigure}
\usepackage{fancyhdr}
%\usepackage{listings}
\usepackage{framed}
\usepackage{graphicx}

\setcounter{MaxMatrixCols}{10}
%TCIDATA{OutputFilter=LATEX.DLL}
%TCIDATA{Version=5.00.0.2570}
%TCIDATA{<META NAME="SaveForMode" CONTENT="1">}
%TCIDATA{LastRevised=Wednesday, February 23, 2011 13:24:34}
%TCIDATA{<META NAME="GraphicsSave" CONTENT="32">}
%TCIDATA{Language=American English}

\pagestyle{fancy}
\setmarginsrb{20mm}{0mm}{20mm}{25mm}{12mm}{11mm}{0mm}{11mm}
\lhead{MA4128} \rhead{Mr. Kevin O'Brien}
\chead{Advanced Data Modelling}
%\input{tcilatex}


% http://www.norusis.com/pdf/SPC_v13.pdf
\begin{document}
	
	
	%SESSION 1: Hierarchical Clustering
	% Hierarchical clustering - dendrograms
	% Divisive vs. agglomerative methods
	% Different linkage methods
	
	%SESSION 2: K-means Clustering
	
	\tableofcontents
	\section{Simple Case Studies}
\subsection{Applications of Cluster Analysis}

We deal with clustering in almost every aspect of daily life. For example, a group of diners sharing the same table in a restaurant may be regarded as a cluster of people. In food stores items of similar nature, such as different types of meat or vegetables are displayed in the same or nearby locations. There is a countless number of examples in which clustering plays an important role. Clustering techniques have been applied to a wide variety of scientific research problems. For example, in the field of medicine, clustering diseases, cures for diseases, or symptoms of diseases can lead to very useful taxonomies. In the field of psychiatry, the correct diagnosis of clusters of symptoms such as paranoia, schizophrenia, etc. is essential for successful therapy. In archeology, researchers have attempted to establish taxonomies of stone tools, funeral objects, etc. by applying cluster analytic techniques. According to the modern system employed in biology, man belongs to the primates, the mammals, the amniotes, the vertebrates, and the animals.

Note how in this classification, the higher the level of aggregation the less similar are the members in the respective class. Man has more in common with all other primates (e.g., apes) than it does with the more "distant" members of the mammals (e.g., dogs), etc.

In general, whenever we need to classify a "mountain" of information into manageable meaningful piles, cluster analysis is of great utility.


%---------------------------------------------------------------%
\subsection{Market Segmentation}
Suppose a market research company wants to undertake direct mail advertising with specific advertisements
for different groups of people. You could use a variety of independent variabless like \textbf{\textit{family income}}
, \textbf{\textit{age}}, \textbf{\textit{number of cars per family}}, \textbf{\textit{number of mobile phones per family}},\textbf{\textit{number of school children per family}}  etc., to see if different postal or zip codes are characterized by particular combinations of demographic variables which could be grouped together to create a better way of directing the mail out.

This firm might in fact find that postal codes could be grouped into a number of clusters, characterized as ``the retirement zone", ``nappy valley", ``the golf club set", the ``rottweiler in a pick-up" district, etc. This sort of grouping might  be valuable in deciding where to place several new wine stores, or `Tummy to Toddler" shops.

Using cluster analysis, a customer ``type" can represent a homogeneous market segment.
Identifying their particular needs in that market allows products to be designed with greater
precision and direct appeal within the segment. Targeting specific segments is cheaper and
more accurate than broad-scale marketing. Customers respond better to segment marketing
which addresses their specific needs, leading to increased market share and customer
retention.

This is valuable, for example, in banking, insurance and tourism markets. Suppose
four clusters or market segments in the vacation travel industry. They are:
\begin{itemize}
	\item[(1)] The high spending elite - they want top level service and expect to be pampered;
	\item[(2)] The escapists - they want to get away and just relax;
	\item[(3)] The educationalist - they want to see new things, go to museums,
	have a safari, or experience new cultures;
	\item[(4)] the sports person - they want the golf course, tennis court, surfing, deep-sea fishing, climbing, etc.
\end{itemize}
Different brochures and advertising is required for each of these.

Brand image analysis, or defining product `types' by customer perceptions, allows
a company to see where its products are positioned in the market relative to those of its
competitors. This type of modelling is valuable for branding new products or identifying
possible gaps in the market. Clustering supermarket products by linked purchasing patterns
can be used to plan store layouts, maximizing spontaneous purchasing opportunities.

\subsection{A Banking example}
Banking institutions have used hierarchical cluster analysis to develop a typology of customers, for two purposes, as follows:
\begin{itemize}
	\item To retain the loyalty of members by designing the best possible new financial products to meet the needs of different groups (clusters), i.e. new product opportunities.
	\item To capture more market share by identifying which existing services are most profitable for which type of customer and improve market penetration.
\end{itemize}
One major bank completed a cluster analysis on a representative sample of its members, according to 16 variables chosen to reflect the characteristics of their financial transaction patterns. From this analysis, 30 types of members were identified. The results were useful for marketing, enabling the bank to focus on products which had the best financial performance; reduce direct mailing costs and increase response rates by targeting product promotions at those customer types most likely to respond; and consequently, to achieve better branding and customer retention.

This facilitated a differential direct advertising of services
and products to the various clusters that differed inter alia by age, income, risk taking levels, and self-perceived financial needs. In this way, the bank could retain and win the business of more profitable customers at lower costs.

%---------------------------------------------------------------------------------%

\end{document}