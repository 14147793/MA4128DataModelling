%-------------------------------------------------------------------------------------------------%
\section{Standardizing the Variables}
If variables are measured on different scales, variables with large values contribute
more to the distance measure than variables with small values. In this example, both
variables are measured on the same scale, so that’s not much of a problem, assuming
the judges use the scales similarly. 

But if you were looking at the distance between two people based on their IQs and incomes in dollars, you would probably find that the
differences in incomes would dominate any distance measures. (A difference of only
\$100 when squared becomes 10,000, while a difference of 30 IQ points would be only
900. I’d go for the IQ points over the dollars!).

Variables that are measured in large numbers will contribute to the distance more than variables recorded in smaller
numbers.
%-------------------------------------------------------------------------------------------------%
In the hierarchical clustering procedure in SPSS, you can standardize variables in
different ways. You can compute standardized scores or divide by just the standard
deviation, range, mean, or maximum. This results in all variables contributing more
equally to the distance measurement. That’s not necessarily always the best strategy,
since variability of a measure can provide useful information. 
%-------------------------------------------------------------------------------------------------%
\newpage

%-------------------------------------------------------------------------------------------------%
\section{Standardizing the Variables}
If variables are measured on different scales, variables with large values contribute
more to the distance measure than variables with small values. In this example, both
variables are measured on the same scale, so that’s not much of a problem, assuming
the judges use the scales similarly. 

But if you were looking at the distance between two people based on their IQs and incomes in dollars, you would probably find that the
differences in incomes would dominate any distance measures. (A difference of only
\$100 when squared becomes 10,000, while a difference of 30 IQ points would be only
900. I’d go for the IQ points over the dollars!).

Variables that are measured in large numbers will contribute to the distance more than variables recorded in smaller
numbers.
%-------------------------------------------------------------------------------------------------%
In the hierarchical clustering procedure in SPSS, you can standardize variables in
different ways. You can compute standardized scores or divide by just the standard
deviation, range, mean, or maximum. This results in all variables contributing more
equally to the distance measurement. That’s not necessarily always the best strategy,
since variability of a measure can provide useful information. 
%-------------------------------------------------------------------------------------------------%
\newpage
\section{Non-Hierarchical Clustering}
This method of clustering is very different from the hierarchical clustering and Ward method, which are applied when there is no prior knowledge of how many clusters there may be or what they are characterized by. The k-means clustering approach is used when you already have hypotheses concerning the number of clusters in your cases or variables. For example, you may want to specify exactly three clusters that are to be as distinct as possible.

This is the type of research question that can be addressed by the k-means clustering algorithm. In general, the k-means method will produce the exact k different clusters demanded of greatest possible distinction. Very often, both the hierarchical and the k-means techniques are used successively.
\begin{itemize}
	\item Ward's method is used to get some sense of the possible number of clusters and the way they merge as seen from the dendrogram.
	\item Then the clustering is rerun with only a chosen optimum number in which to place all
	the cases (i.e. k means clustering).
\end{itemize}
%-------------------------------------------------------------------------------------------------%
In these methods the desired number of clusters is specified in advance and the `best' solution
is chosen. The steps in such a method are as follows:
\begin{itemize}
	\item[1] Choose initial cluster centres (essentially this is a set of observations that are far apart
	— each subject forms a cluster of one and its centre is the value of the variables for
	that subject).
	\item[2] Assign each subject to its `nearest' cluster, defined in terms of the distance to the
	centroid.
	\item[3] Find the centroids of the clusters that have been formed
	\item[4] Re-calculate the distance from each subject to each centroid and move observations that
	are not in the cluster that they are closest to.
	\item[5] Continue until the centroids remain relatively stable.
\end{itemize}
%-------------------------------------------------------------------------------------------------%
Non-hierarchical cluster analysis tends to be used when large data sets are involved. It is
sometimes preferred because it allows subjects to move from one cluster to another (this is
not possible in hierarchical cluster analysis where a subject, once assigned, cannot move to a
different cluster). Two disadvantages of non-hierarchical cluster analysis are: 
\begin{itemize}
	\item[1]it is often
	diffcult to know how many clusters you are likely to have and therefore the analysis may have
	to be repeated several times 
	\item[2] it can be very sensitive to the choice of initial cluster centres. Again, it may be worth trying di?erent ones to see what impact this has.
\end{itemize}
%-------------------------------------------------------------------------------------------------%
\subsection{Optimal Number of Clusters}
One of the biggest problems with cluster analysis is identifying the optimum number of
clusters. As the joining process continues, increasingly dissimilar clusters must be joined. i.e. the classification becomes increasingly artificial. Deciding upon the optimum number
of clusters is largely subjective, although looking at a dendrogram would help.

%-------------------------------------------------------------------------------------------------%
\subsection{Optimal Number of Clusters}
One of the biggest problems with cluster analysis is identifying the optimum number of
clusters. As the joining process continues, increasingly dissimilar clusters must be joined. i.e. the classification becomes increasingly artificial. Deciding upon the optimum number
of clusters is largely subjective, although looking at a dendrogram would help.
