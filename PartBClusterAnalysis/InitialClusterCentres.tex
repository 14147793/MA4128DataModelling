%------------------------------------------------------------------%
\subsection{Initial Cluster Centres}
The first step in k-means clustering is finding the k centres. This is done iteratively. 
You start with an initial set of centres and then modify them until the change between two iterations is small enough.

If you have good guesses for the centres, you can use those
as initial starting points; otherwise, you can let SPSS find k cases that are well separated and use these values 
as initial cluster centers. (i.e. The clustering process starts by randomly assigning objects to a number of
clusters).

%------------------------------------------------------------------%
\subsection{Initial Cluster Centres}
The first step in k-means clustering is finding the k centres. This is done iteratively. 
You start with an initial set of centres and then modify them until the change between two iterations is small enough.

If you have good guesses for the centres, you can use those
as initial starting points; otherwise, you can let SPSS find k cases that are well separated and use these values 
as initial cluster centers. (i.e. The clustering process starts by randomly assigning objects to a number of
clusters).


%------------------------------------------------------------------%

K-means clustering is very sensitive to outliers, since they will usually be selected as initial cluster centers. 
This will result in outliers forming clusters with small numbers of cases. Before you start a cluster analysis, 
screen the data for outliers and remove them from the initial analysis. The solution may also depend on the order 
of the cases in the data.

After the initial cluster centers have been selected, each case is assigned to the closest
cluster, based on its distance from the cluster centers. After all of the cases have been
assigned to clusters, the cluster centers are recomputed, based on all of the cases in the
cluster.

The cases are then successively reassigned to other clusters to minimize the within-cluster variation, 
which is basically the (squared) distance from each observation to the center of the associated cluster. 
If the reallocation of an case to another cluster decreases the within-cluster variation, this case is reassigned
to that cluster.

Case assignment is done again, using these updated cluster centers. You keep
assigning cases and recomputing the cluster centers until no cluster center changes
appreciably or the maximum number of iterations (10 by default) is reached.
\newpage
%------------------------------------------------------------------%

