\documentclass[a4paper,12pt]{article}
%%%%%%%%%%%%%%%%%%%%%%%%%%%%%%%%%%%%%%%%%%%%%%%%%%%%%%%%%%%%%%%%%%%%%%%%%%%%%%%%%%%%%%%%%%%%%%%%%%%%%%%%%%%%%%%%%%%%%%%%%%%%%%%%%%%%%%%%%%%%%%%%%%%%%%%%%%%%%%%%%%%%%%%%%%%%%%%%%%%%%%%%%%%%%%%%%%%%%%%%%%%%%%%%%%%%%%%%%%%%%%%%%%%%%%%%%%%%%%%%%%%%%%%%%%%%
\usepackage{eurosym}
\usepackage{vmargin}
\usepackage{amsmath}
\usepackage{graphics}
\usepackage{epsfig}
\usepackage{subfigure}
\usepackage{fancyhdr}
%\usepackage{listings}
\usepackage{framed}
\usepackage{graphicx}

\setcounter{MaxMatrixCols}{10}
% http://nyx-www.informatik.uni-bremen.de/664/1/smith_tr_02.pdf

\pagestyle{fancy}
\setmarginsrb{20mm}{0mm}{20mm}{25mm}{12mm}{11mm}{0mm}{11mm}
\lhead{Dublin \texttt{R}} \rhead{24 April 2013}
\chead{Principal Components Analysis with \texttt{R}}

\begin{document}

\section{Principal Components}

The \texttt{princomp()} function produces an \textit{unrotated} principal component analysis.

\begin{framed}
\begin{verbatim}
# Pricipal Components Analysis
# entering raw data and extracting PCs from the correlation matrix 
fit <- princomp(mydata, cor=TRUE)
summary(fit) # print variance accounted for 
loadings(fit) # pc loadings 
plot(fit,type="lines") # scree plot 
fit$scores # the principal components
biplot(fit)
\end{verbatim}
\end{framed}

\subsection{The \texttt{principal()} function(psych package) }
The \texttt{principal()} function in the \textbf{psych }package can be used to extract and rotate principal components.
\begin{framed}
\begin{verbatim}
# Varimax Rotated Principal Components
# retaining 5 components 
library(psych)
fit <- principal(mydata, nfactors=5, rotate="varimax")
fit # print results
\end{verbatim}
\end{framed}
%-------------------------------------------------------------- %
\subsection{Scree Plots}
\texttt{screeplot.default} plots the variances against the number of the principal component. This is also the plot method for classes "\texttt{princomp}" and "\texttt{prcomp}".


\end{document}
