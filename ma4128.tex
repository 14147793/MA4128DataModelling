% !TEX TS-program = pdflatex
% !TEX encoding = UTF-8 Unicode

% This is a simple template for a LaTeX document using the "article" class.
% See "book", "report", "letter" for other types of document.

\documentclass[11pt]{article} % use larger type; default would be 10pt

\usepackage[utf8]{inputenc} % set input encoding (not needed with XeLaTeX)

%%% Examples of Article customizations
% These packages are optional, depending whether you want the features they provide.
% See the LaTeX Companion or other references for full information.

%%% PAGE DIMENSIONS
\usepackage{geometry} % to change the page dimensions
\geometry{a4paper} % or letterpaper (US) or a5paper or....
% \geometry{margin=2in} % for example, change the margins to 2 inches all round
% \geometry{landscape} % set up the page for landscape
%   read geometry.pdf for detailed page layout information

\usepackage{graphicx} % support the \includegraphics command and options

% \usepackage[parfill]{parskip} % Activate to begin paragraphs with an empty line rather than an indent

%%% PACKAGES
\usepackage{booktabs} % for much better looking tables
\usepackage{array} % for better arrays (eg matrices) in maths
\usepackage{paralist} % very flexible & customisable lists (eg. enumerate/itemize, etc.)
\usepackage{verbatim} % adds environment for commenting out blocks of text & for better verbatim
\usepackage{subfig} % make it possible to include more than one captioned figure/table in a single float
% These packages are all incorporated in the memoir class to one degree or another...

%%% HEADERS & FOOTERS
\usepackage{fancyhdr} % This should be set AFTER setting up the page geometry
\pagestyle{fancy} % options: empty , plain , fancy
\renewcommand{\headrulewidth}{0pt} % customise the layout...
\lhead{}\chead{}\rhead{}
\lfoot{}\cfoot{\thepage}\rfoot{}

%%% SECTION TITLE APPEARANCE
\usepackage{sectsty}
\allsectionsfont{\sffamily\mdseries\upshape} % (See the fntguide.pdf for font help)
% (This matches ConTeXt defaults)

%%% ToC (table of contents) APPEARANCE
\usepackage[nottoc,notlof,notlot]{tocbibind} % Put the bibliography in the ToC
\usepackage[titles,subfigure]{tocloft} % Alter the style of the Table of Contents
\renewcommand{\cftsecfont}{\rmfamily\mdseries\upshape}
\renewcommand{\cftsecpagefont}{\rmfamily\mdseries\upshape} % No bold!

%%% END Article customizations

%%% The "real" document content comes below...

\title{MA4128 : Advanced Data Modelling}
\author{Kevin O'Brien}
\date{} % Activate to display a given date or no date (if empty),
         % otherwise the current date is printed 

\begin{document}
\maketitle
%-------------------------------------------------------------------------%
\section*{Section 1 Dimensionality Reduction}
Principal Component Analysis and Factor Analysis\\

(Principal Components (PCs) and Factors are essentially the same thing.)


\subsection*{1.d}
Rotation Methods\\
Simplify Interpretation of PCA by "spreading more evenly" the explanation across multiple PCs\\
Varimax is the most commonly used of the orthogonal procedures. There is also the oblique rotation procedures.

\subsection*{1.n}
True structure of data is contained in some unobserved latent variables. These variables are unknown and are explored after the analysis is carried out (this is Factor Analysis - Structure Detection etc). \\
We dont know how many Latent variables there are. Use PCA to make a guess : build articifical PCs to represent these latent variables.\\

Importantly PCA is a very specific mathematical technique - used to estimate number of PCs. Factor Analysis includes this technique, but is broader in scope and what it sets out to do.
%-------------------------------------------------------------------------%
\section*{Section 2 Cluster Analysis}


\subsection*{2.c}

Dendrogram and verticle icicle plot.\\

Can assess appropriate number of clusters by when clusters get linked to each other (early or late in the schedule). Still a very subjective decision. No "right" answer.\\


%-------------------------------------------------------------------------%
\section*{Section 3 Linear Model}

\subsection*{3.c} For two candidate models, one with lowest AIC is the preferred model.

\subsection*{3.h}
Variance Inflation Factor and Tolerance

\[VIF = \frac{1}{\mbox{tolerance}}\]

\subsection*{3.i}
Multicollinearity: Inflates the standard errors of the regression estimates.(i.e. very wide confidence intervals, and strange inaccurate p-values)
Multicollinearity : Reduces predictive power of the model. ( Multicollinearity is indicative of 
overfitting)

\subsection*{3.j}
Find best set of independent variables. Considering overfitting and Law of Parsimony.

\subsection*{3.k} Variable selection procedures are used to determine which set of independent variables best describes the data. The variables are inserted or omitted according to the strength of their correlation with the response variable
Describe
\begin{itemize}
\item Forward Selection
\item Backward Selection
\item Stepwise Selection
\end{itemize}

Can use the AIC as the method of determining improvement in model. Stepwise is a combination of the first two.


%-------------------------------------------------------------------------%
\section*{Section 4: Logistic Regression}

\subsection*{4.C}
The key aspect of this question is the nature of the response variable.
With OLS regression, the outcome variable is assumed to be normally distributed continious variable ( for example Height).
With Logistic Regression the outcome variable is a binary value (either 0 or 1/ Success or Failure), essentially categorical, and by definition such an outcome is not normally distributed.




%-------------------------------------------------------------------------%
\section*{Section 5: Missing Data}



\subsection*{5.E}
Traditional techniques: Casewise deletion.
If a case (set of observations) contains a missing value for some variable, then this case is discounted from the analysis.

We have seen in class a data set that contained approx 1000 observations, but only 300 or so were used to construct the model. Massively undermining the strenght of the model.


\subsection*{5.F}
Missing Data can massively reduced the amount of knowledge that can be obtained from a data set.

%-------------------------------------------------------------------------%
\section*{Section 6 MANOVA and Discriminant Analysis}

\subsection*{6.g}

Discriminant Analysis requires a lot of assumption to be met in order to be a valid analysis.
Multiinomial Logistic Regression can provide the same the type of analysis, but require less assumptions to be met.

\subsection*{6.h}
Use of Training/Validation/testing procedures.


\subsection*{6.i}

Example of a univariate outlier 85 in the following data set
\[ \{  14,6,12,14,9,11,15,7,15,85\} \]

Bivariate outlier: unusual combination of values, even when the individual values are not unusual themselves
Example: 6 foot tall person (1.83 meters) in height would not be unusal. A person weighing eight stone would not be unusual eithe, but a 6 foot tall person weighing 8 stone would be very unusual.

Mahalanobis Distance takes covariance (correlation) into account when computing distance from centre of a cluster of values. A value that does not fit in with the overall trend in the data would have a high mahalanobis distance.

Short person 8 stone - small Mahalanobis distance\\
Tall Person 17 stone - small Mahalanobis Distance\\
Tall Person - 8 stone - very high Mahalanobis Distance\\

%-------------------------------------------------------------------------%
\end{document}
