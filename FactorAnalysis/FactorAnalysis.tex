%-----------------------------------------%
\subsection{Factor Analysis}

\begin{itemize}
\item Factor Analysis is commonly used in Psychological and Sociological studies (hence the appropriate examples in text).
\item In factor analysis, the factor is used instead of principal component. Essentially they are the same thing.
\end{itemize}

\subsection{Purposes of factor analysis}

There are two main purposes or applications of factor analysis:

\begin{itemize}
\item[1.] \textbf{Data Reduction}
Reducing data to a smaller set of summary variables e.g., psychological questionnaires often aim to measure several psychological constructs, with each latent variable measured using multiple items which can be combined in a smaller number of factor scores.
\item[2.] \textbf{Exploring theoretical structure}
Theoretical questions about the underlying structure of psychological phenomena can be explored and empirically tested using factor analysis e.g., is intelligence better described as a single, general factor, or as consisting of multiple, independent dimensions?
\end{itemize}

%-----------------------------------------------------------------------%

\subsection{Types of Factor Analysis}
There are two main types of extraction:
\begin{description}
\item[Principal components (PC):] Analyses all variance in the items, usually preferred when the goal is to reduce a set of variables down to a smaller number of factors and to create composite scores for these factors for use in subsequent analysis.
\item[Principal axis factoring (PAF):] Analyses shared variance amongst the items. Used more often for theoretical explorations of the underlying factor structure.
\end{description}
The most widely used method in factor analysis is the PAF method.

\subsection{Factor Loading}
The Factor loading is Correlation between a variable and a factor (or principal component), and the key to understanding the nature of a particular factor. Squared factor loadings indicate what percentage of the variance in an original variable is explained by a factor.

\subsection{Factor Rotation}
Factor rotation is the process of adjusting the factor axes to achieve a simpler and pragmatically more meaningful factor solution - the goal is a simple factor structure. The most commonly used type of rotation is \emph{\textbf{varimax}}. Others include equimax.

\textbf{Orthogonal factor rotation} is factor rotation such that their axes are maintained at 90 degrees (i.e. orthogonal). Each factor is independent of, or orthogonal to, all other factors. The correlation between the factors is determined to be zero. The opposite of this is \textbf{Oblique factor rotation}, rotation such that the extracted factors are allowed to be correlated
%-----------------------------------------------------------------------%
