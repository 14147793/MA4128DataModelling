\documentclass[a4]{beamer}
\usepackage{amssymb}
\usepackage{graphicx}
\usepackage{subfigure}
\usepackage{newlfont}
\usepackage{multicol}
\usepackage{amsmath,amsthm,amsfonts}
%\usepackage{beamerthemesplit}
\usepackage{pgf,pgfarrows,pgfnodes,pgfautomata,pgfheaps,pgfshade}
\usepackage{mathptmx} % Font Family
\usepackage{helvet} % Font Family
\usepackage{color}
\mode<presentation> {
\usetheme{Default} % was Frankfurt
\useinnertheme{rounded}
\useoutertheme{infolines}
\usefonttheme{serif}
%\usecolortheme{wolverine}
% \usecolortheme{rose}
\usefonttheme{structurebold}
}
\setbeamercovered{dynamic}
\title[MA4413]{Statistics for Computing \\ {\normalsize MA4413 Lecture 10A}}
\author[Kevin O'Brien]{Kevin O'Brien \\ {\scriptsize Kevin.obrien@ul.ie}}
\date{Autumn Semester 2013}
\institute[Maths \& Stats]{Dept. of Mathematics \& Statistics, \\ University \textit{of} Limerick}

\renewcommand{\arraystretch}{1.5}
%----------------------------------------------------------------------------------------------------------%
\begin{document}

\begin{frame}
\titlepage
\end{frame}
\begin{frame}
	\frametitle{Binary Classification}
	Recall the possible outcomes of a hypothesis test procedure. In particular recall the two important types of error. Importantly the binary classification prediction procedure can yield wrong predictions.
	\begin{center}
		\begin{tabular}{|c|c|c|} \hline
			& Null hypothesis & Null hypothesis   \\
			& ($H_0$) true	 & ($H_0$) false \\ \hline
			Reject 	       & \textbf{Type I error }   & \textbf{Correct outcome} \\
			null hypothesis& False positive  & True positive \\ \hline
			Fail to reject & \textbf{Correct outcome} & \textbf{Type II error} \\
			null hypothesis & True negative  & False negative \\ \hline
		\end{tabular} 
	\end{center}
\end{frame}
%------------------------------------------------------%
\begin{frame}
	\frametitle{Accuracy, Precision and Recall}
	Let us simplify the last table, and present it in the context of a binary prediction procedure.
	\begin{center}
		\begin{tabular}{|c|c|c|}
			\hline  & Predicted Negative & Predicted Positive \\ 
			\hline Observed Negative & True Negative & False Positive \\ 
			\hline Observed Positive & False Negative & True Positive \\ 
			\hline 
		\end{tabular} 
	\end{center}
\end{frame}
%------------------------------------------------------%
\begin{frame}
	\frametitle{Accuracy, Precision and Recall}
	Important metrics for determining how usefulness of the prediction procedure are : Accuracy, Recall and Precision.
	\\
	Accuracy, Precision and recall are defined as
	\[\mbox{Accuracy}=\frac{tp+tn}{tp+tn+fp+fn} \]
	\[\mbox{Precision}=\frac{tp}{tp+fp} \] 
	\[\mbox{Recall}=\frac{tp}{tp+fn} \]
\end{frame}
%------------------------------------------------------%
\begin{frame}
	\frametitle{Accuracy, Precision and Recall}
	Another measure is the F-measure.
	The F measure is computed as
	\[F = 2 \cdot \frac{\mathrm{precision} \cdot \mathrm{recall}}{ \mathrm{precision} + \mathrm{recall}}\]
	
\end{frame}
%------------------------------------------------------%
\begin{frame}
	\frametitle{Questions}
	\begin{center}
		\begin{tabular}{|c|c|c|}
			\hline
			& Predicted  & Predicted \\
			& Negative & Positive \\ \hline
			Negative Cases & TN: 9,700  & FP: 165 \\ \hline
			Positive Cases & FN: 35 & TP: 100 \\ \hline
		\end{tabular} 
	\end{center}
\end{frame}
%------------------------------------------------------%
\begin{frame}
	\frametitle{Accuracy, Precision and Recall}
	
	With reference to the table above, compute each of the following appraisal metrics.
	\begin{multicols}{2} 
		\begin{itemize}
			\item[a.] Accuracy
			\item[b.] Precision
			\item[c.] Recall
			\item[d.] $F$ measure
		\end{itemize}
	\end{multicols}
\end{frame}
%------------------------------------------------------%
\begin{frame}
	\frametitle{Accuracy, Precision and Recall}
	\begin{itemize}
		\item Why is the accuracy value so high?
		\item Why is the F-measure so low?
		\item This is the class-imbalance problem: more``negative" outcomes which skews the statistic, but these outcomes are the least relevant.
		\item F-measure disregards the irrelevant ``true negatives, and concentrates on the more relevant potential outcomes.
	\end{itemize}
\end{frame}
%------------------------------------------------------%
\begin{frame}
	BLANK
\end{frame}
%------------------------------------------------------%
\begin{frame}
	BLANK
\end{frame}
%------------------------------------------------------%
\begin{frame}
	BLANK
\end{frame}
%------------------------------------------------------%
\begin{frame}
	BLANK
\end{frame}
%------------------------------------------------------%
\begin{frame}
	BLANK
\end{frame}


%---------------------------- %

\end{document}