\section{Training and validation}
%http://www.jmp.com/support/help/Validation_2.shtml
Using Validation and Test Data

%When you have sufficient data, you can subdivide your data into three parts called the training, validation, and test data. During the selection process, models are fit on the training data, and the prediction error for the models so obtained is found by using the validation data. This prediction error on the validation data can be used to decide when to terminate the selection process or to decide what effects to include as the selection process proceeds. Finally, once a selected model has been obtained, the test set can be used to assess how the selected model generalizes on data that played no role in selecting the model.

In some cases you might want to use only training and test data. For example, you might decide to use an information criterion to decide what effects to include and when to terminate the selection process. In this case no validation data are required, but test data can still be useful in assessing the predictive performance of the selected model. In other cases you might decide to use validation data during the selection process but forgo assessing the selected model on test data. 
%Hastie, Tibshirani, and Friedman (2001) note that it is difficult to give a general rule on how many observations you should assign to each role. They note that a typical split might be 50\% for training and 25% each for validation and testing.



%---------------------------%
\section{Supervised learning}


Supervised learning is the machine learning task of inferring a function from supervised training data. The training data consist of a set of training examples. In supervised learning, each example is a pair consisting of an input object (typically a vector) and a desired output value (also called the supervisory signal). A supervised learning algorithm analyzes the training data and produces an inferred function, which is called a classifier (if the output is discrete, see classification) or a regression function (if the output is continuous, see regression). The inferred function should predict the correct output value for any valid input object. This requires the learning algorithm to generalize from the training data to unseen situations 

in a "reasonable" way (see inductive bias). 


