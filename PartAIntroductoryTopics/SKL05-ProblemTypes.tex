\documentclass[SKL-MASTER.tex]{subfiles}
\begin{document}
\LARGE
\noindent \textbf{Machine learning: the problem setting}

\noindent In general, a learning problem considers a set of $n$ samples of data and try to predict properties of unknown data. If
each sample is more than a single number, and for instance a multi-dimensional entry (aka multivariate data), is it said
to have several variables, also known as attributes or \textbf{\textit{features}}.\\
\bigskip
We can separate learning problems in a few large categories:

\begin{itemize}
\item \textbf{Supervised learning}, in which the data comes with additional attributes that we want to predict.\\ 
%(Click here to go to the Scikit-Learn supervised learning page).
\bigskip
This problem can be either:
\begin{description}
\item[Classification:] samples belong to two or more classes and we want to learn from already labeled data how
to predict the class of unlabeled data. \\ An example of classification problem would be the digit recognition
example, in which the aim is to assign each input vector to one of a finite number of discrete categories.
\item[Regression:] if the desired output consists of one or more continuous variables, then the task is called
regression. \\ An example of a regression problem would be the prediction of the weight of a pony as a
function of its age and height.
\end{description}

\newpage
\item  \textbf{Unsupervised learning}, in which the training data consists of a set of input vectors $x$ without any corresponding
target values. \\ \bigskip The goal in such problems may be
\begin{itemize}
\item to discover groups of similar examples within the data, where
it is called \textbf{\textit{clustering}}, \item to determine the distribution of data within the input space, known as \textbf{\textit{density estimation}},
\item to project the data from a high-dimensional space down to two or thee dimensions for the purpose of
visualization 
\end{itemize}
% (Click here to go to the Scikit-Learn unsupervised learning page).
\end{itemize}

%======================================================================== %
\end{document}