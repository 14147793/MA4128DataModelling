\documentclass[]{article}

%opening
\title{}
\author{}

\begin{document}

\maketitle

\begin{abstract}

\end{abstract}

%-------------------------------------------------------------------------%
\section*{Section 3 Linear Model}

\subsection*{3.c} For two candidate models, one with lowest AIC is the preferred model.

\subsection*{3.h}
Variance Inflation Factor and Tolerance

\[VIF = \frac{1}{\mbox{tolerance}}\]

\subsection*{3.i}
Multicollinearity: Inflates the standard errors of the regression estimates.(i.e. very wide confidence intervals, and strange inaccurate p-values)
Multicollinearity : Reduces predictive power of the model. ( Multicollinearity is indicative of 
overfitting)

\subsection*{3.j}
Find best set of independent variables. Considering overfitting and Law of Parsimony.

\subsection*{3.k} Variable selection procedures are used to determine which set of independent variables best describes the data. The variables are inserted or omitted according to the strength of their correlation with the response variable
Describe
\begin{itemize}
\item Forward Selection
\item Backward Selection
\item Stepwise Selection
\end{itemize}

Can use the AIC as the method of determining improvement in model. Stepwise is a combination of the first two.



\end{document}
