\subsection{Residuals diagnostics in mixed models}

%schabenberger
The marginal and conditional means in the linear mixed model are
$E[\boldsymbol{Y}] = \boldsymbol{X}\boldsymbol{\beta}$ and
$E[\boldsymbol{Y|\boldsymbol{u}}] = \boldsymbol{X}\boldsymbol{\beta} + \boldsymbol{Z}\boldsymbol{u}$, respectively.

A residual is the difference between an observed quantity and its estimated or predicted value. In the mixed
model you can distinguish marginal residuals $r_m$ and conditional residuals $r_c$. 


%\subsection{Marginal and Conditional Residuals}
%
%A marginal residual is the difference between the observed data and the estimated (marginal) mean, $r_{mi} = y_i - x_0^{\prime} \hat{b}$
%A conditional residual is the difference between the observed data and the predicted value of the observation,
%$r_{ci} = y_i - x_i^{\prime} \hat{b} - z_i^{\prime} \hat{\gamma}$
%
%In linear mixed effects models, diagnostic techniques may consider `conditional' residuals. A conditional residual is the difference between an observed value $y_{i}$ and the conditional predicted value $\hat{y}_{i} $.
%
%\[ \hat{epsilon}_{i} = y_{i} - \hat{y}_{i} = y_{i} - ( X_{i}\hat{beta} + Z_{i}\hat{b}_{i}) \]
%
%However, using conditional residuals for diagnostics presents difficulties, as they tend to be correlated and their variances may be different for different subgroups, which can lead to erroneous conclusions.

%1.5
%http://support.sas.com/documentation/cdl/en/statug/63033/HTML/default/viewer.htm#statug_mixed_sect024.htm






\begin{equation}
r_{mi}=x^{T}_{i}\hat{\beta}
\end{equation}

\subsection{Marginal Residuals}
\begin{eqnarray}
\hat{\beta} &=& (X^{T}R^{-1}X)^{-1}X^{T}R^{-1}Y \nonumber \\
&=& BY \nonumber
\end{eqnarray}


\newpage