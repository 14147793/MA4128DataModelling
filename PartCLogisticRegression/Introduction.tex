\documentclass[12pt]{article}

\usepackage{amsmath}
\usepackage{amssymb}
\usepackage{framed}
%opening

\begin{document}
\tableofcontents
\newpage
\section{Binomial Logistic Regression} 
A binomial logistic regression (often referred to simply as logistic regression), predicts the probability that an observation falls into one of two categories of a \textbf{dichotomous} dependent variable based on one or more independent variables that can be either continuous or categorical.

\subsection{Applications}
For example, you could use binomial logistic regression to understand whether exam performance can be predicted based on revision time, test anxiety and lecture attendance (i.e., where the dependent variable is ``exam performance", measured on a dichotomous scale – ``passed" or ``failed" – and you have three independent variables: ``revision time", ``test anxiety" and ``lecture attendance"). 


Alternately, you could use binomial logistic regression to understand whether drug use can be predicted based on prior criminal convictions, drug use amongst friends, income, age and gender (i.e., where the dependent variable is ``drug use", measured on a dichotomous scale – ``yes" or ``no" – and you have five independent variables: ``prior criminal convictions", ``drug use amongst friends", ``income", ``age" and "gender").
\newpage
\subsection{Assumptions}

\begin{description}
\item[Assumption 1:] Your dependent variable should be measured on a \textbf{dichotomous scale}. Examples of dichotomous variables include gender (two groups: "males" and "females"), presence of heart disease (two groups: "yes" and "no"), personality type (two groups: "introversion" or "extroversion"), body composition (two groups: "obese" or "not obese"), and so forth. \\
\newline
However, if your dependent variable was not measured on a dichotomous scale, but a continuous scale instead, you will need to carry out \textbf{multiple regression}, whereas if your dependent variable was measured on an ordinal scale, \textbf{ordinal regression} would be a more appropriate starting point.

\item[Assumption 2:] You have one or more independent variables, which can be either continuous (i.e., an interval or ratio variable) or categorical (i.e., an ordinal or nominal variable). 



\item[Assumption 3:] You should have independence of observations and the dependent variable should have mutually exclusive and exhaustive categories.

\item[Assumption 4:] There needs to be a linear relationship between any continuous independent variables and the logit transformation of the dependent variable. 
\end{description}
\begin{framed}
\newpage
\subsubsection*{Types of Variables (Revision)}
\begin{itemize}
\item Examples of continuous variables include revision time (measured in hours), intelligence (measured using IQ score), exam performance (measured from 0 to 100), weight (measured in kg), and so forth. 

\item Examples of ordinal variables include \textit{Likert} items (e.g., a 7-point scale from "strongly agree" through to "strongly disagree"), amongst other ways of ranking categories (e.g., a 3-point scale explaining how much a customer liked a product, ranging from "Not very much" to "Yes, a lot"). 
\item Examples of nominal variables include gender (e.g., 2 groups: male and female), ethnicity (e.g., 3 groups: Caucasian, African American and Hispanic), profession (e.g., 5 groups: surgeon, doctor, nurse, dentist, therapist), and so forth.
\end{itemize}
\end{framed}
%----------------------------------------------------------------------------------------%

%In our enhanced binomial logistic regression guide, we show you how to: 
%(a) make a natural log transformation of your dependent variable; 
%(b) use the Box-Tidwell (1962) procedure to test for linearity; 
%(c) interpret the SPSS output from this test and report the results.
\end{document}
