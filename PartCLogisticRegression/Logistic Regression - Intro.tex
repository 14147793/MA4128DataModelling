\documentclass[]{article}


\begin{document}
\tableofcontents

\section{Introduction to Logistic Regression}
Logistic regression or logit regression is a type of probabilistic statistical classification model.[1] It is also used to predict a binary response from a binary predictor, used for predicting the outcome of a categorical dependent variable (i.e., a class label) based on one or more predictor variables (features). 

That is, it is used in estimating empirical values of the parameters in a qualitative response model. The probabilities describing the possible outcomes of a single trial are modeled, as a function of the explanatory (predictor) variables, using a logistic function. 

Logistic regression, also called a logit model, is used to model dichotomous (i.e. Binary) outcome variables. In the logit model the log odds of the outcome is modeled as a linear combination of the predictor variables.

\subsection{Examples of Logistic Regression}

\begin{description}
\item[Example 1:]  Suppose that we are interested in the factors that influence whether a political candidate wins an election.  The outcome (response) variable is binary (0/1); \textit{ win or lose}.  The predictor variables of interest are the amount of money spent on the campaign, the amount of time spent campaigning negatively and whether or not the candidate is an incumbent.

\item[Example 2:]  A researcher is interested in how variables, such as GRE (Graduate Record Exam scores), GPA (grade point average) and prestige of the undergraduate institution, effect admission into graduate school. The response variable, \textit{admit/don't admit}, is a binary variable.
\end{description}


\end{document}
