\section{Multinomial Logistic Regression}
\subsection{Multinomial Logistic Regression}
Multinomial Logistic Regression is useful for situations in which you want to be able to classify
subjects based on values of a set of predictor variables. This type of regression is similar to logistic
regression, but it is more general because the dependent variable is not restricted to two categories.
For Example, In order to market films more effectively, movie studios want to predict what type of
film a moviegoer is likely to see. By performing a Multinomial Logistic Regression, the studio
can determine the strength of influence a person’s age, gender, and dating status has upon the type
of film they prefer. The studio can then slant the advertising campaign of a particular movie
toward a group of people likely to go see it.
%-------------------------------------------------------------------------------------%
\newpage

Examples of multinomial logistic regression

\begin{description}
	\item[Example 1.] People's occupational choices might be influenced by their parents' occupations and their own education level. We can study the relationship of one's occupation choice with education level and father's occupation.  The occupational choices will be the outcome variable which consists of categories of occupations.
	
	\item[Example 2.] A biologist may be interested in food choices that alligators make. Adult alligators might have difference preference than young ones. The outcome variable here will be the types of food, and the predictor variables might be the length of the alligators and other environmental variables.
	
	\item[Example 3.] Entering high school students make program choices among general program, vocational program and academic program. Their choice might be modeled using their writing score and their social economic status.
\end{description}