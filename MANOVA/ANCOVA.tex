\documentclass[a4paper,12pt]{article}
%%%%%%%%%%%%%%%%%%%%%%%%%%%%%%%%%%%%%%%%%%%%%%%%%%%%%%%%%%%%%%%%%%%%%%%%%%%%%%%%%%%%%%%%%%%%%%%%%%%%%%%%%%%%%%%%%%%%%%%%%%%%%%%%%%%%%%%%%%%%%%%%%%%%%%%%%%%%%%%%%%%%%%%%%%%%%%%%%%%%%%%%%%%%%%%%%%%%%%%%%%%%%%%%%%%%%%%%%%%%%%%%%%%%%%%%%%%%%%%%%%%%%%%%%%%%
\usepackage{eurosym}
\usepackage{vmargin}
\usepackage{amsmath}
\usepackage{graphics}
\usepackage{epsfig}
\usepackage{subfigure}
\usepackage{fancyhdr}
\usepackage{listings}
\usepackage{framed}
\usepackage{graphicx}
\usepackage{amsmath}
\usepackage{chngpage}
%\usepackage{bigints}


\setcounter{MaxMatrixCols}{10}
%TCIDATA{OutputFilter=LATEX.DLL}
%TCIDATA{Version=5.00.0.2570}
%TCIDATA{<META NAME="SaveForMode" CONTENT="1">}
%TCIDATA{LastRevised=Wednesday, February 23, 2011 13:24:34}
%TCIDATA{<META NAME="GraphicsSave" CONTENT="32">}
%TCIDATA{Language=American English}

%\pagestyle{fancy}
%\setmarginsrb{20mm}{0mm}{20mm}{25mm}{12mm}{11mm}{0mm}{11mm}
%\lhead{MA4413} \rhead{Mr. Kevin O'Brien}
%\chead{Statistics For Computing}
%\input{tcilatex}

\begin{document}
\begin{center}
       \includegraphics[scale=0.55]{shieldtransparent2}
\end{center}

\begin{center}
\vspace{1cm}
\large \bf {FACULTY OF SCIENCE AND ENGINEERING} \\[0.5cm]
\normalsize DEPARTMENT OF MATHEMATICS AND STATISTICS \\[1.25cm]
\large \bf {MID-SEMESTER ASSESSMENT} \\[1.5cm]
\end{center}

\begin{tabular}{ll}
MODULE CODE: MA4128 & SEMESTER: Spring \\[1cm]
MODULE TITLE: Advanced Data Modeling & DURATION OF EXAM: 1 hour \\[1cm]
LECTURER: Mr. Kevin O'Brien & GRADING SCHEME: 100 marks \\
& \phantom{GRADING SCHEME:} \footnotesize {20\% of module grade} \\[0.8cm]
\\[1cm]
\end{tabular}
\begin{center}
{\bf INSTRUCTIONS TO CANDIDATES}
\end{center}

{\noindent \\ Scientific calculators approved by the University of Limerick can be used. \\
Formula sheet and statistical tables provided at the end of the exam paper.\\
Students must attempt any 4 questions from 5.}
\newpage



% - Section 1 Inference Procedures
        % a. Parametric
        % b. Non Parametric
% - Section 2 Linear Models
        % a. SLR
        % b. MLR
% - Section 3 Linear Models
        % a. Robust Regression
        % b. AIC
% - Section 4 Statistical Process control
        % a. Control Limits
        % b. Theory Questions
        % c. Interpreting Charts
        % d. CUSUM and ARL
% - Section 5 Experimental Design 1
        % a. Definitions for ED
        % b. One Way ANOVA
% - Section 6 Experimental Design 2
        % a.
        % b.


\subsection*{Question 1. (10 marks) Distriutional Assumptions}
\begin{itemize}
\item[(a)] (10 Marks) 

The data set \texttt{X} and \texttt{Y} are both assumed to be normally distributed. The Shapiro-Wilk test was carried out to assess whether or not this assumption is valid for data set \texttt{X}.
\begin{itemize}
	\item[i.] (1 marks) Formally state the null and alternative hypothesis.
	\item[ii.] (2 marks) What is your conclusion for this procedure? Justify your answer.
\end{itemize}
\begin{framed}
\begin{verbatim}
> shapiro.test(X)
	
Shapiro-Wilk normality test
data:  X
W = 0.9292, p-value = 0.372
\end{verbatim}
\end{framed}

\item [(c)] The data set \texttt{X} and \texttt{Y} are both assumed to be normally distributed. A graphical procedure was carried out to assess whether or not this assumption of normality is valid for data set \texttt{Y}. Consider the Q-Q plot in the figure below.

\begin{center}
	\includegraphics[scale=0.55]{Q5examQQplot}
\end{center}

\begin{itemize}
	\item[i.] (2 marks) Provide a brief description on how to interpret this plot.
	\item[ii.] (1 marks) What is your conclusion for this procedure? Justify your answer.
\end{itemize}

\item[(b)] The typing speeds for one group of 12 Engineering students were recorded both at the beginning of year 1 of their studies. The results (in words per minute) are given below:

\begin{center}
	\begin{tabular}{|c|c|c|c|c|}
		\hline
		% Subject& A& B& C& D& E &F &G &H \\ \hline
		149  & 146 & 122 & 142 &  153\\ \hline
		137 & 161 & 156&   170&  159
		\\ \hline
	\end{tabular}
\end{center}
Use the Dixon Q-test to determine if there is an outlier present in this data. You may assume a significance level of 5\%.
\begin{itemize}
	\item[(i.)](1 Mark)	State the null and alternative Hypothesis for this test.
	\item[(ii.)](2 Marks) Compute the test statistic
	\item[(iii.)](1 Mark) State the appropriate critical value.
	\item[(iv.)](1 Mark) What is your conclusion to this procedure.
\end{itemize}
\end{itemize} 
%--------------------------------------------------------------------------------------------------------%



\subsection*{Question 2. (10 marks) Binary Classification }
%\subsection*{Question 4. (20 marks) }

\begin{itemize}
	\item[(a)] \textbf{\textit{Binary Classification (6 Marks)}}\\
	For following binary classification outcome table, calculate the following appraisal metrics.
	\begin{itemize}	
		\item[(i)] (1 Mark)	accuracy;
		\item[(ii)] (1 Mark)	recall;
		\item[(iii)] (1 Mark)	precision;
		\item[(iv)] (1 Mark)	F-measure.
	\end{itemize}	
	
	\begin{center}
		\begin{tabular}{|c|c|c|}
			\hline  & \phantom{spa}Predict Negative\phantom{spa} & \phantom{spa}Predict Positive\phantom{spa} \\ 
			\hline\phantom{spa} Observed Negative \phantom{spa}&	9530	&	10	\\ 
			\hline \phantom{spa}Observed Positive\phantom{spa} & 	300	&	160	\\ 
			\hline 
		\end{tabular} 
	\end{center}
	
	\begin{itemize}	
		\item[(v)] (2 Marks) Explain why the F-measure is considered a more informative measure of performance than the Accuracy score.
			\item[(iii.)](2 Marks) Define Specificity and Sensitivity. You make reference to previous answers.
			\item[(iv.)](3 Marks) What is a ROC curve? Explain its function, how it is determined, and the means of interpreting the curve. Support your answer with a sketch.
	\end{itemize}

\end{itemize}


%-------------------------------Start of Question 2A%
\newpage


%-------------------------------End  of Question 2A%
%-------------------------------Start of Question 2B%
%\item[(b)](6 marks)
%The scatter-plot contains the regression line for the fitted model. Three diagnostic plots, used to assess the suitability of the fitted model, are presented on the following pages. Provide a brief interpretation for each of the three diagnostic plots described in part(a). The scatter-plot for the data is also presented.

%\begin{center}
%\includegraphics[scale=0.60]{ExamQ2plot2}
%\end{center}
%\newpage
%\begin{center}
%\includegraphics[scale=0.55]{ExamQ2diag1}
%\end{center}
%
%\begin{center}
%\includegraphics[scale=0.55]{ExamQ2diag2}
%\end{center}







\newpage









%%-------------------------------------%
%\subsection*{Question 3. (20 marks) Multiple Linear Regression Models }
%\begin{itemize}
%
%\item[(a)] Explain the following terms:
%\begin{itemize}
%\item[i.] (2 marks) Over-fitting,
%\item[ii.] (2 marks) Multicollinearity,
%\item[iii.] (2 Marks) Heteroscedascity.
%\end{itemize}
%\item[(b)] Answer the following questions related to model selection techniques for linear models.
%\begin{itemize}
%\item[i.] (2 marks) Explain why the adjusted $R^2$ value may differ in value from the corresponding multiple $R^2$ value for the same fitted model.
%\item[ii.] (2 marks) Explain how the \emph{Akaike information criterion} would used to compare two models fitted for the same data.
%\end{itemize}
%
%\item[(c)]
%In an experiment to determine hydrolysable tannins in plants by absorption spectroscopy, the following results from ten samples were obtained and are tabulated below. A simple linear regression model, predicting absorbance values using concentration as the independent variable, was fitted to the data.
%
%
%%Absorbance= c(0.084, 0.183, 0.326, 0.464, 0.643, 0.707, 0.717, 0.734 ,0.749 ,0.732) ;
%%Concentration= c(0.123, 0.288, 0.562, 0.921, 1.420, 1.717, 1.921, 2.137 ,2.321, 2.467) ;
%%plot(Concentration,Absorbance,pch=18,col="red",font.axis=2,font.lab=2)
%%abline(coef(lm(Absorbance~Concentration)))
%%
%%Conc.Squared = (Concentration^2)
%%Conc.Cubed = (Concentration^3)
%%ModelA = lm(Absorbance~Concentration)
%%ModelB = lm(Absorbance~Concentration+Conc.Squared)
%%ModelC = lm(Absorbance~Concentration+Conc.Squared+Conc.Cubed)
%
%\begin{center}
%\begin{tabular}{|c||c|c|c|c|c|}
%  \hline
%  % after \\: \hline or \cline{col1-col2} \cline{col3-col4} ...
%Sample & 1 & 2 & 3 & 4 & 5 \\ \hline
%Absorbance & 0.084& 0.183& 0.326& 0.464& 0.643\\
%Concentration & 0.123& 0.288& 0.562& 0.921& 1.420\\ \hline
%Sample & 6 & 7 & 8 & 9 & 10 \\ \hline
%Absorbance & 0.707& 0.717& 0.734 &0.749 &0.732\\
%Concentration & 1.717& 1.921& 2.137 &2.321&2.467\\
%  \hline
%\end{tabular}
%\end{center}
%
%\begin{center}
%\includegraphics[scale=0.55]{ExamQ3plot}
%\end{center}
%
%
%\begin{itemize}
%\item[i.] (2 marks) Is the simple linear regression model approach suitable for this study? Explain your answer with reference to the scatter-plot.
%%\item[ii] (2 marks) Two polynomial models were also fitted to the data. Description of all three fitted models are found in the three blocks of \texttt{R} code below. The \emph{Akaike information criterion} is also listed, for each of the three fitted models.
%\end{itemize}
%
%
%
%\item[(d)] 
%
%%\end{document}
%--------------------------------------------------------%
\newpage
\subsection*{Question 3. (10 marks) Hierarchical Clustering}


\begin{itemize}
	\item[i.](2 Marks) What is the purpose of a cluster analysis?
	
	\item[ii.](2 Marks)  A discriminant analysis is similar to a cluster analysis; however, there is one fundamental difference.  Explain this difference.
	
	\item[iii.](2 Marks)  What is the difference between a linkage method and a distance measure?
	
	\item[iv.](2 Marks)  Compare and contrast any two linkage methods.
	
	\item 
	
	Why do you standardize variables before carrying out a cluster analysis.
	
	Explain why using the standardized value may not be suitable in some cases?
	
	\item[v.](2 Marks)  How do we determine the appropriate number of clusters?  Give two different visualization methods that are used to display the outcome of a cluster analysis.
	
	\item[vi.](2 Marks) Standardization
	
	\item[vii.](2 Marks)  Explain the difference between Ward's method and k-means
	clustering.
	%\item[7.] Vertical Icicle Plot

	\item[viii.](2 Marks)  In the context of hierarchical cluster analysis, distinguish between agglomerative clustering and divisive clustering.
	\item[ix.](2 Marks)  What is a vertical icicle plot used for? Give a brief description, supporting your answer with sketches.
	\item[x.](2 Marks)  Compute the Euclidean distance between the following points.
	\[ A = \c(4,6,8,2)\]
	\[ B = \c(3,6,1,6)\]
	
\end{itemize}



%-----------------------------------------------------------%
\newpage

\subsection*{Question 4. (10 marks) K-Means Clustering}



\begin{figure}[h!]
\centering
\includegraphics[width=1.1\linewidth]{ANOVA}
\caption{}
\label{fig:ANOVA}
\end{figure}


%-----------------------------------------------------------------%
\newpage

\subsection*{Question 5. (10 marks) Modelling Count Variables }


\begin{itemize}
	\item[(i)] 
	
	\item[(ii)]
	
	\item[(iii)] 
	
	\item[(iv)] 
	
	\item[(v)] 
\end{itemize}




%------------------------------------------------------------------------ %
\Large{
\newpage
	\section*{Formulas and Tables}
	
	\subsection*{Critical Values for Dixon Q Test}
	{
		\Large
		\begin{center}
			\begin{tabular}{|c|c|c|c|}
				\hline  N  & $\alpha=0.10$  & $\alpha=0.05$  & $\alpha=0.01$  \\ 
			    &{\normalsize \textit{Confidence}$=0.90$ } & {\normalsize \textit{Confidence}$=0.95$ }  & {\normalsize \textit{Confidence}$=0.99$ }   \\ \hline
				3  & 0.941 & 0.97  & 0.994 \\ \hline
				4  & 0.765 & 0.829 & 0.926 \\ \hline
				5  & 0.642 & 0.71  & 0.821 \\ \hline
				6  & 0.56  & 0.625 & 0.74  \\ \hline
				7  & 0.507 & 0.568 & 0.68  \\ \hline
				8  & 0.468 & 0.526 & 0.634 \\ \hline
				9  & 0.437 & 0.493 & 0.598 \\ \hline
				10 & 0.412 & 0.466 & 0.568 \\ \hline
				11 & 0.392 & 0.444 & 0.542 \\ \hline
				12 & 0.376 & 0.426 & 0.522 \\ \hline
				13 & 0.361 & 0.41  & 0.503 \\ \hline
				14 & 0.349 & 0.396 & 0.488 \\ \hline
				15 & 0.338 & 0.384 & 0.475 \\ \hline
			\phantom{sp}	16 \phantom{sp} & 0.329 & 0.374 & 0.463 \\ \hline
			\end{tabular} 
		\end{center}
	}
	


\end{document}




