\documentclass[a4paper,12pt]{article}
%%%%%%%%%%%%%%%%%%%%%%%%%%%%%%%%%%%%%%%%%%%%%%%%%%%%%%%%%%%%%%%%%%%%%%%%%%%%%%%%%%%%%%%%%%%%%%%%%%%%%%%%%%%%%%%%%%%%%%%%%%%%%%%%%%%%%%%%%%%%%%%%%%%%%%%%%%%%%%%%%%%%%%%%%%%%%%%%%%%%%%%%%%%%%%%%%%%%%%%%%%%%%%%%%%%%%%%%%%%%%%%%%%%%%%%%%%%%%%%%%%%%%%%%%%%%
\usepackage{eurosym}
\usepackage{vmargin}
\usepackage{amsmath}
\usepackage{graphics}
\usepackage{epsfig}
\usepackage{framed}
\usepackage{subfigure}
\usepackage{fancyhdr}

\setcounter{MaxMatrixCols}{10}
%TCIDATA{OutputFilter=LATEX.DLL}
%TCIDATA{Version=5.00.0.2570}
%TCIDATA{<META NAME="SaveForMode"CONTENT="1">}
%TCIDATA{LastRevised=Wednesday, February 23, 201113:24:34}
%TCIDATA{<META NAME="GraphicsSave" CONTENT="32">}
%TCIDATA{Language=American English}

\pagestyle{fancy}
\setmarginsrb{20mm}{0mm}{20mm}{25mm}{12mm}{11mm}{0mm}{11mm}
\lhead{MA4128} \rhead{Kevin O'Brien} \chead{One-way MANOVA}%\input{tcilatex}

%http://www.electronics.dit.ie/staff/ysemenova/Opto2/CO_IntroLab.pdf
\begin{document}

\section{Underlying Assumptions of MANOVA}
When you choose to analyze your data using a one-way MANOVA, part of the process involves checking to make sure that the data you want to analyze can actually be analyzed using a one-way MANOVA. You need to do this because it is only appropriate to use a one-way MANOVA if your data ``passes" nine assumptions that are required for a one-way MANOVA to give you a valid result. \\ Do not be surprised if, when analyzing your own data using SPSS, one or more of these assumptions is violated. This is not uncommon when working with real-world data.\\ However, even when your data fails certain assumptions, there is often a solution to overcome this.

In practice, checking for these nine Assumptions adds some more time to your analysis, requiring you to work through additional procedures in SPSS when performing your analysis, as well as thinking a little bit more about your data. These nine assumptions are presented below:

\begin{itemize}
	\item \textbf{Assumption 1}: Your two or more dependent variables should be measured at the interval or ratio level (i.e., they are continuous). Examples of variables that meet this criterion include revision time (measured in hours), intelligence (measured using IQ score), exam performance (measured from 0 to 100), weight (measured in kg), and so forth.\\
	
	\item \textbf{Assumption 2}: Your independent variable should consist of two or more categorical, independent groups. Example independent variables that meet this criterion include ethnicity (e.g., 3 groups: Caucasian, Afro-Caribbean and South Asian), physical activity level (e.g., 4 groups: sedentary, low, moderate and high), profession (e.g., 5 groups: surgeon, doctor, nurse, dentist, therapist), and so forth.\\
	
	\item \textbf{Assumption 3}: You should have independence of observations, which means that there is no relationship between the observations in each group or between the groups themselves. For example, there must be different participants in each group with no participant being in more than one group. This is more of a study design issue than something you can test for, but it is an important assumption of the one-way MANOVA.\\
	
	\item \textbf{Assumption 4}: You should have an adequate sample size. Although the larger your sample size, the better; for MANOVA, you need to have more cases in each group than the number of dependent variables you are analyzing.\\
	
	\item \textbf{Assumption 5}: There are no univariate or multivariate outliers. First, there can be no (univariate) outliers in each group of the independent variable for any of the dependent variables.\\ This is a similar assumption to the one-way ANOVA, but for each dependent variable that you have in your MANOVA analysis. \\ Univariate outliers are often just called outliers and are the same type of outliers you will have come across if you have conducted t-tests or ANOVAs.\\ We refer to them as univariate in this guide to distinguish them from multivariate outliers. Multivariate outliers are cases which have an unusual combination of scores on the dependent variables.	
	\begin{framed}
Approaches for detecting outliers
\begin{itemize}
\item[(1)] detect univariate outliers using boxplots,  
\item[(2)] check for multivariate outliers using a measure called \textbf{Mahalanobis distance}.
\end{itemize} 
	\end{framed}
	\bigskip
	\item \textbf{Assumption 6}: There is multivariate normality. Unfortunately, multivariate normality is a particularly tricky assumption to test for and cannot be directly tested in SPSS. Instead, normality of each of the dependent variables for each of the groups of the independent variable is often used in its place as a best 'guess' as to whether there is multivariate normality.\\ You can test for this using the \textbf{\textit{Shapiro-Wilk test of normality}}, which is easily tested for using SPSS. \\
	
	\item \textbf{Assumption 7}: There is a linear relationship between each pair of dependent variables for each group of the independent variable. If the variables are not linearly related, the power of the test is reduced. \\ You can test for this  assumption by plotting a scatterplot matrix for each group of the independent variable. In order to do this, you will need to split your data file in SPSS before generating the scatterplot matrices.
	\item \textbf{Assumption 8}: There is a homogeneity of variance-covariance matrices. You can test this Assumption in SPSS using \textbf{Box's M test} of equality of covariance. If your data fails this Assumption, you may also need to use SPSS to carry out \textbf{Levene's test} of homogeneity of variance to determine where the problem may lie.
	\item \textbf{Assumption 9}: There is no multicollinearity. Ideally, you want your dependent variables to be moderately correlated with each other. If the correlations are low, you might be better off running separate one-way ANOVAs, and if the correlation(s) are too high (greater than 0.9), you could have multicollinearity. This is problematic for MANOVA and needs to be screened out.
\end{itemize}
\subsection{Testing Assumptions with SPSS}
You can check assumptions 5, 6, 7, 8 and 9 using SPSS. Before doing this, you should make sure that your data meets Assumptions 1, 2, 3 and 4, although you don't need SPSS to do this. Just remember that if you do not run the statistical tests on these assumptions correctly, the results you get when running a one-way MANOVA might not be valid.
\newpage

\end{document}