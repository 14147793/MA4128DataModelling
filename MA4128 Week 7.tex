\documentclass[a4paper,12pt]{article}
%%%%%%%%%%%%%%%%%%%%%%%%%%%%%%%%%%%%%%%%%%%%%%%%%%%%%%%%%%%%%%%%%%%%%%%%%%%%%%%%%%%%%%%%%%%%%%%%%%%%%%%%%%%%%%%%%%%%%%%%%%%%%%%%%%%%%%%%%%%%%%%%%%%%%%%%%%%%%%%%%%%%%%%%%%%%%%%%%%%%%%%%%%%%%%%%%%%%%%%%%%%%%%%%%%%%%%%%%%%%%%%%%%%%%%%%%%%%%%%%%%%%%%%%%%%%
\usepackage{eurosym}
\usepackage{vmargin}
\usepackage{amsmath}
\usepackage{graphics}
\usepackage{epsfig}
\usepackage{subfigure}
\usepackage{fancyhdr}

\setcounter{MaxMatrixCols}{10}
%TCIDATA{OutputFilter=LATEX.DLL}
%TCIDATA{Version=5.00.0.2570}
%TCIDATA{<META NAME="SaveForMode"CONTENT="1">}
%TCIDATA{LastRevised=Wednesday, February 23, 201113:24:34}
%TCIDATA{<META NAME="GraphicsSave" CONTENT="32">}
%TCIDATA{Language=American English}

\pagestyle{fancy}
\setmarginsrb{20mm}{0mm}{20mm}{25mm}{12mm}{11mm}{0mm}{11mm}
\lhead{MA4128} \rhead{Kevin O'Brien} \chead{Week 6} %\input{tcilatex}

\begin{document}

\tableofcontents
\section{Agenda for Today's Class}

\begin{itemize}
	\item Review of Important Topics
	\item Review of K-Means Clustering (SPSS Exercise)
	\item Two-Step Clustering
	\item Review of Regression (Optional for Math Science Students)
\end{itemize}

\section{Important Topics}

\begin{itemize}
	\item \textbf{Multi-collinearity}: Multi-collinearity occurs when two or more predictors in the model are
	correlated and provide redundant information about the response. Examples of pairs of multi-collinear predictors are years of education and income, height and weight of a person, and assessed value and square footage
	of a house.
	
	\item \textbf{Consequences of high multicollinearity}:
	Multi-collinearity leads to decreased reliability and predictive power of statistical models, and hence, very often, confusing and misleading results.
	\item Multicollinearity will be dealt with in a future component of this course: Variable Selection Procedures.
\end{itemize}

\end{document}