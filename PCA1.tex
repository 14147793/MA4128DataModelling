
\documentclass[12pt]{article}

\usepackage{framed}
\usepackage{amsmath}
\usepackage{amssymb}
\usepackage{graphics}
\usepackage{graphicx}
%opening
\title{Multivariate Statistics}
%\author{MA4605}
%http://www.qualityamerica.com/Knowledgecenter/statisticalprocesscontrol/interpreting_process_capability.asp
\begin{document}
\section*{Principal Component Analysis}

\subsection*{Variable Reduction Procedure}

Principal component analysis is a variable reduction procedure. It is useful when you have
obtained data on a number of variables (possibly a large number of variables), and believe that
there is some redundancy in those variables. In this case, redundancy means that some of the
variables are correlated with one another, possibly because they are measuring the same
construct. Because of this redundancy, you believe that it should be possible to reduce the
observed variables into a smaller number of principal components (artificial variables) that will
account for most of the variance in the observed variables.

\subsection*{Characteristics of Principal Components}
The first component extracted in a principal
component analysis accounts for a maximal amount of total variance in the observed variables.
Under typical conditions, this means that the first component will be correlated with at least
some of the observed variables. It may be correlated with many.

The second component extracted will have two important characteristics. First, this component
will account for a maximal amount of variance in the data set that was not accounted for by the
first component. Again under typical conditions, this means that the second component will be
correlated with some of the observed variables that did not display strong correlations with
component 1.

The second characteristic of the second component is that it will be uncorrelated with the first
component. Literally, if you were to compute the correlation between components 1 and 2, that
correlation would be zero.

The remaining components that are extracted in the analysis display the same two characteristics:
each component accounts for a maximal amount of variance in the observed variables that was
not accounted for by the preceding components, and is uncorrelated with all of the preceding
components. A principal component analysis proceeds in this fashion, with each new component
accounting for progressively smaller and smaller amounts of variance (this is why only the first
few components are usually retained and interpreted). When the analysis is complete, the
resulting components will display varying degrees of correlation with the observed variables, but
are completely uncorrelated with one another.

\end{document}