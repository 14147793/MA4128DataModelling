\documentclass[]{article}

%opening
\title{}
\author{}

\begin{document}

\maketitle

\title{MA4128 Missing Data}

\section{Missing Data}

\subsection{Imputation}

\subsection{Missing Not At Random (MNAR)}

\subsection{Missin Complete}

%-------------------------------------------------------%
\subsection{Complete Data}

Complete data means that the value of each sample unit is observed or known. Complete data is much easier to work with than censored data, and basic statistical analysis techniques assume that we have complete data.


\subsection{Censored Data}

There are three types of possible censoring schemes, right censored data (also called suspended data), interval censored data, and left censored data.



Right Censored (Suspended)

These are data for which we know only its minimum value. In reliability testing, for example, not all of the tested units will necessarily fail within the testing period. Then all we know is that the failure time exceeds the testing time. In microbiology, there is a practical threshold above which we cannot count colonies on a Petri dish. In sequential sifting, we known only the minimum diameter of the largest particles that don't pass through the first sieve. This type of data is commonly called right-censored or suspended data.


\subsubsection{Interval Censored}

These are data for which we know only that they lie between a certain minimum and maximum. Interval censoring arises commonly when we assign measurements into categories or intervals. For example, a survey may ask people which income range they have, and offer several contiguous intervals, rather than ask their exact income. In reliability testing, for example, we may only be inspecting the units every T hours, so can only record that a unit failed between nT and (n+1)T hours. This is sometimes called inspection data.


Left Censored

These are data for which we know only its maximum value. In scientific experiments, for example, we may not be able to measure some quantity because it is below the threshold of detection (e.g. chemical concentration).


\section{Missing Data}


\subsection{Missing completely at random}
There are several reasons why the data may be missing. They may be missing because equipment malfunctioned, the weather was terrible, or people got sick, or the data were not entered correctly. Here the data are missing completely at random (MCAR). When we say that data are missing completely at random, we mean that the probability that an observation (Xi) is missing is unrelated to the value of Xi or to the value of any other variables. Thus data on family income would not be considered MCAR if people with low incomes were less likely to report their family income than people with higher incomes. Similarly, if Whites were more likely to omit reporting income than African Americans, we again would not have data that were MCAR because ``missingness" would be correlated with ethnicity. However if a participant's data were missing because he was stopped for a traffic violation and missed the data collection session, his data would presumably be missing completely at random. Another way to think of MCAR is to note that in that case any piece of data is just as likely to be missing as any other piece of data.

Notice that it is the value of the observation, and not its "missingness," that is important. If people who refused to report personal income were also likely to refuse to report family income, the data could still be considered MCAR, so long as neither of these had any relation to the income value itself. This is an important consideration, because when a data set consists of responses to several survey instruments, someone who did not complete the Beck Depression Inventory would be missing all BDI subscores, but that would not affect whether the data can be classed as MCAR.

This nice feature of data that are MCAR is that the analysis remains unbiased. We may lose power for our design, but the estimated parameters are not biased by the absence of data.

\subsection{Missing at random}
Often data are not missing completely at random, but they may be classifiable as \textbf{missing at random} (MAR). For data to be missing completely at random, the probability that Xi is missing is unrelated to the value of Xi or other variables in the analysis. But the data can be considered as missing at random if the data meet the requirement that ``missingness" does not depend on the value of $X_i$ after controlling for another variable.

For example, people who are depressed might be less inclined to report their income, and thus reported income will be related to depression. Depressed people might also have a lower income in general, and thus when we have a high rate of missing data among depressed individuals, the existing mean income might be lower than it would be without missing data. However, if, within depressed patients the probability of reported income was unrelated to income level, then the data would be considered MAR, though not MCAR.

The phraseology is a bit awkward here because we tend to think of randomness as not producing bias, and thus might well think that Missing at Random is not a problem. Unfortunately is is a problem, although in this case we have ways of dealing with the issue so as to produce meaningful and relatively unbiased estimates. But just because a variable is MAR does not mean that you can just forget about the problem.

\subsection{Missing Not at random}
If data are not missing at random or completely at random then they are classed as \textbf{Missing Not at Random} (MNAR). For example, if we are studying mental health and people who have been diagnosed as depressed are less likely than others to report their mental status, the data are not missing at random. Clearly the mean mental status score for the available data will not be an unbiased estimate of the mean that we would have obtained with complete data. The same thing happens when people with low income are less likely to report their income on a data collection form.

When we have data that are MNAR we have a problem. The only way to obtain an unbiased estimate of parameters is to model missingness. In other words we would need to write a model that accounts for the missing data. That model could then be incorporated into a more complex model for estimating missing values. This is not a task anyone would take on lightly. See Dunning and Freedman (2008) for an example.


\end{document}
